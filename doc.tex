%% Waehle Sprache
%\def\MyLanguage{english}
\def\MyLanguage{ngerman}
%% bt: Bachelor Thesis
%% mt: Master Thesis
\documentclass[\MyLanguage,bt]{dbvdoc}

% Eingabe-Loesungen:
% - Default: use utf8 on your computer
% - use an editor capable of converting character-sets and editing utf8 files
%   on a latin1 system (some versions of vi do)
% - use \"{a}, \"{o}, \"{u}, \ss{} instead of non-ascii characters

\usepackage[utf8]{inputenc}
%%\usepackage[utf8,latin1]{inputenc}  %% Alternative Eingabe
\usepackage{babel}
\usepackage[babel,autostyle,german=quotes]{csquotes}
\usepackage[backend=biber,maxnames=99,language=\MyLanguage,bibstyle=alphabetic,sortcites=true,labelalpha=true]{biblatex}
\usepackage{hyperref}
\usepackage{breakurl}
%\usepackage[hyphenbreaks]{breakurl}  %% Falls in \url auch bei Bindestrich getrennt werden soll. Weitere Optionen: RTFM
\usepackage{graphicx}
\usepackage{epsfig}
\usepackage{subfigure}
\usepackage{psfrag}
\usepackage{color}

% \setcounter{lotdepth}{2}

% ++ es werden keine underfull hboxes als Fehler ausgegeben,
%    da das ja nur heißt, dass die Seite noch nicht ganz voll ist
\hbadness=10000
\clubpenalty=10000 % schliesst Schusterjungen aus
\widowpenalty=10000 % schliesst Hurenkinder aus

\pagenumbering{roman}

%\bibliographystyle{galpha2a}
\renewcommand*{\labelalphaothers}{\textsuperscript{}} %% plus-zeichen abschalten
\bibliography{doc}
\addbibresource{doc.bib}
\addbibresource{lit.bib}

%%%%%%%%%%%%%%
%%% MACROS %%%
%%%%%%%%%%%%%%
%% if macros shall be used, put them into a separate file 'macros.tex'
%%\input{macros}

%%%%%%%%%%%%%%%%%%%%
%%% Worttrennung %%%
%%%%%%%%%%%%%%%%%%%%
%% if hyphenation patterns are needed, put them into a separate file 'hyph.tex'
%%\input{hyph}


%%%%%%%%%%%%%%
%\setcounter{tocdepth}{5}
%\setcounter{secnumdepth}{5}

\sloppy		%% avoid writing over linebreak


\begin{document}
\clearpage
  \selectlanguage{\MyLanguage}
  \begin{deckblatt}
    \Titel{Titel Arbeit}
    \Name{Mustermann}
    \Vorname{Matthias}
    \Wohnort{Musterstadt}
    \BetreuerA{Erstpr\"{u}fer: Prof.~Dr.-Ing.~Frank Deinzer}
    \BetreuerB{Zweitpr\"{u}ferin: Prof.~Dr.-Ing.~Martina Meier}
    \Ende{1. YYY 20YY}
    \Fach{Informatik}  % for Master: \Fach{Informationssysteme}
    \Schwerpunkt{Medieninformatik}
    \Angefertigt{Angefertigt an der Fakultät für Informatik und Wirtschaftsinformatik der Technischen Hochschule Würzburg-Schweinfurt/bei der Firma XYZ}
  \end{deckblatt}
\clearpage
\mbox{}
\vfill
\begin{center}
\ifpdf
	\includegraphics[width=6cm]{qrcode-thesis.png}
\else
	\includegraphics[width=6cm]{qrcode-thesis.eps}
\fi
\end{center}
\clearpage

\noindent Hiermit versichere ich, dass ich die vorgelegte Bachelorarbeit/Masterarbeit selbstständig verfasst und noch nicht
anderweitig zu Prüfungszwecken vorgelegt habe. Alle benutzten Quellen und Hilfsmittel sind
angegeben, wörtliche und sinngemäße Zitate wurden als solche gekennzeichnet.\\[5mm]
Würzburg, den\\[20mm]
(Unterschrift)

\vfill

\noindent Hiermit willige ich ein, dass zum Zwecke der Überprüfung auf Plagiate meine vorgelegte Arbeit in
digitaler Form an PlagScan übermittelt und diese vorübergehend (max. 5 Jahre)
in der von PlagScan geführten Datenbank gespeichert wird, sowie persönliche Daten, die Teil dieser
Arbeit sind, dort hinterlegt werden.

Die Einwilligung ist freiwillig. Ohne diese Einwilligung kann unter Entfernung aller persönlichen
Angaben und Wahrung der urheberrechtlichen Vorgaben die Plagiatsüberprüfung nicht verhindert
werden. Die Einwilligung zur Speicherung und Verwendung der persönlichen Daten kann jederzeit
durch Erklärung gegenüber der Fakultät widerrufen werden.\\[5mm]
Würzburg, den\\[20mm]
(ggf. Unterschrift)

\clearpage

\begin{center}
\bf Übersicht
\end{center}
TEXT DEUTSCH


\vfill
\begin{center}
\bf Abstract
\end{center}
TEXT ENGLISCH

\vfill
\cleardoublepage

\tableofcontents

\cleardoublepage \pagenumbering{arabic}

%%%%%%%%%%%%%%%%%%%%%%%
%%% Inlucde chapter %%%
%%%%%%%%%%%%%%%%%%%%%%%
\include{doc01}   
\cleardoublepage
%\include{doc02}   
%\cleardoublepage
%% usw.
\nocite{*}  %% Das ist nur zu Demo-Zwecken da!
%%%%%%%%%%%%%%
%%% Anhang %%%
%%%%%%%%%%%%%%
%\begin{appendix}
%\include{doc-a0} 
%\cleardoublepage
%\include{doc-a1} 
%\cleardoublepage
%%usw...
%\end{appendix}

%% Literatur
\addcontentsline{toc}{chapter}{\bibname}
\printbibliography
\cleardoublepage

%% Bilderverzeichnis
\addcontentsline{toc}{chapter}{\listfigurename}
\listoffigures\cleardoublepage

%% Tabellenverzeichnis
\addcontentsline{toc}{chapter}{\listtablename}
\listoftables\cleardoublepage


\end{document}
